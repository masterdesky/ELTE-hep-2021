\section{Introduction} \label{sec:1}
Through heavy-ion physics experiments -- especially at the Relativistic Heavy Ion Collider (RHIC) -- over the past few years it was observed, that in case of non-central heavy-ion collisions, a specific momentum space anisotropy so called \emph{elliptic flow} is created. In these experiments mostly Au$+$Au collisions were observed and analysed (see eg. \cite{Lin2002} or \cite{Collaboration2009}).

This anisotropy can be observed through eg. the measurement of the angle distribution of outgoing particles and serves as a strong evidence for the existence of quark-gluon plasma (QGP), a dense form of matter with exotic properties that completely filled the universe in the very first moments after the Big Bang. The formation of QGP strongly affects, how the initial anisotropy during the collision is transferred to the final, observed state. Measuring the exact characteristics of this final \q{freeze-out} thus could give us precise insights about the early state of this anisotropy and the transport properties of the QGP too.

In a non-central heavy-ion collision only a fraction of the valence- and sea quarks participate in the actual collision itself. The so called \q{spectators} that do not fall into the participation zone are continues their travel down the pipe of the collider. Inside the zone an \emph{elliptic} or almond shaped volume of QGP is created that thermalize extremely quickly and scatter particles around every direction. Since the results for the same circumstances (colliding material, energy, etc.) are governed by the orientation of the colliding beam lines, these measurements are usually categorized by their \emph{centrality} \citep{Snellings2011}.

One of the main interest for studies of the elliptic flow is the $n=2$ azimuthal anisotropy coefficient $v_{2}$. This is the second-order coefficient in the Fourier expansion of the distribution of the outgoing particles and can be defined as

\begin{equation} \label{eq:1.1}
	\Df{N}{\phi}
	=
	v_{0}
	\left[
		1
		+
		2 v_{1} \cos \left( \phi \right)
		+
		2 v_{2} \cos \left( 2 \phi \right)
	\right],
\end{equation}
where

\begin{equation} \label{eq:1.2}
	\phi = \phi_{0} - \Psi,
\end{equation}
where $\phi_{0}$ is the raw angle of the particle track, while $\Psi$ is the angle of the reaction plane. In this project the determination of this quantity were in focus.